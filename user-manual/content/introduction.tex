\chapter{Einführung}
\large\textbf{Willkommen zu \lectStudio{}: Digitalisieren Sie Ihre Vorträge mit Leichtigkeit!}
\\\\
\lectStudio{} ist eine freie und kostenlose Software zur Präsentation von PDF-basierten Vorlesungen, die eine intuitive und effiziente Bedienbarkeit bietet. Eine Vielzahl von interaktiven Werkzeugen, wie Stift-basierte Annotationen mit digitaler Tinte, werden durch \lectStudio{} bereitgestellt. Durch interaktive Feedback-Funktionen, wie Quizze und Nachrichten über Web-Browser, lassen Sie die Zuhöher aktiv am Vortrag teilnehmen. Zu den wesentlichen Funktionen von \lectStudio{} gehören auch die Aufzeichnung, das Echtzeit-Streaming und die Nachbearbeitung von Vorträgen.
\\
Für die Netzwerkfunktionen sind keine zusätzlichen Server-Installatio-nen sowie externe Server notwendig, es wird aber die Möglichkeit geboten, eine eigene externe und modulare Infrastruktur zu nutzen.
\\\\
Dieses Benutzerhandbuch richtet sich an Lehrende ohne technische Vorkenntnisse und bietet einen Überblick über die Funktionen der Anwendungen und enthält schrittweise Anweisungen zum Ausführen einer Vielzahl von Aufgaben.


\subsection*{Motivation}
Für eTeaching werden für verschiedene Aufgaben verschiedene Werkzeuge eingesetzt. Im Folgenden sind die Aufgaben und Werkzeuge beispielhaft aufgelistet.

\begin{itemize}
	\item \textbf{Folien präsentieren}
	\begin{itemize}
		\item Z.B. PowerPoint, PDF Viewer, ...
	\end{itemize}

	\item \textbf{Vorlesung aufzeichnen und nachbearbeiten}
	\begin{itemize}
		\item Z.B. Camtasia, Snagit
	\end{itemize}

	\item \textbf{Umfragen}
	\begin{itemize}
		\item Z.B. Pingo
	\end{itemize}

	\item \textbf{Streaming: Videokonferenzen und Webinare}
	\begin{itemize}
		\item Z.B. Zoom
	\end{itemize}
\end{itemize}

\vspace{0.5cm}

\lectStudio{} bietet eine integrierte Alternative zu den im Einsatz befindlichen Werkzeugen. Einen Vergleich der Funktionen zeigt \autoref{tab:tool-comparison}.

\vspace{1.0cm}
{
\footnotesize
\begin{tabularx}{\textwidth}{llllll}\toprule
	\textbf{Funktion}	& \textbf{\color{steelblue}\lectStudio}	& \textbf{Camtasia}	& \textbf{Snagit}	& \textbf{Zoom}	&	\textbf{Pingo}	\\ \midrule
	Präsentieren		& \makecell{PDF / \\Whiteboard}		& \makecell{Bildschirm / \\ Ausschnitt}	& \makecell{Bildschirm / \\Ausschnitt}	& \makecell{Bildschirm / \\Whiteboard}	& 	\\ \hline
	Aufzeichnen			& Vektorformat		& Bildschirm	& Bildschirm	& Bildschirm	& 	\\ \hline
	\makecell{Kamera \\aufzeichnen}	& 		& \faCheck{}	& \faCheck{}	& \faCheck{}	& 	\\ \hline
	Nachbearbeiten		& \faCheck{}		& \faCheck{}	& \faCheck{}	& 	& 	\\ \hline
	\makecell{Nachbearbeiten \\mit zus. Medien}	& 	& \makecell{Bilder, \\Videos, ...}	& Annotationen	& 	& 	\\ \hline
	Streamen			& \makecell{Multicast / \\WebRTC (dev)}	& 	& 	& Webkonferenz	& 	\\ \hline
	Chat				& Unidirektional		& 	& 	& \makecell{Gruppenchat / \\Bidirektional}	& 	\\ \hline
	Umfragen			& \faCheck{}		& 	& 	& \faCheck{}	& \faCheck{}	\\ \hline
	Plattform			& \makecell{\faWindows{} \faApple{} \faLinux{} \\Web-Client}	& \faWindows{} \faApple{}	& \faWindows{} \faApple{}	& Web	& Web	\\ \hline
	Videokompression	& ca. 0,5 MB/min		& \makecell{ca. 1,5 MB/min \\ \\ Mit Kamera\\ca. 3 MB/min}	& ca. 1,5 MB/min	& N / A	& 	\\ \hline
	Open Source			& In Vorbereitung	& 	& 	& 	& Client	\\ \hline
	Kostenlos			& \faCheck{}		& 	& 	& \makecell{(\faCheck{}) Gruppen-\\meetings\\max. 40 Min}	& \faCheck{}	\\ \hline
	\makecell{Benutzerdefiniert \\erweiterbar}	& \faCheck{}	& 	& 	& 	& 	\\ \bottomrule
	\hline
\end{tabularx}
}
\captionof{table}{Vergleich der Funktionen von eTeaching-Werkzeugen}\label{tab:tool-comparison}



\subsection*{Zusätzliche Hilfe erhalten}
Um technischen Support und Software-Unterstützung zu erhalten, wenden Sie sich bitte an \href{mailto:lecturestudio@esa.tu-darmstadt.de}{\textbf{lecturestudio@esa.tu-darmstadt.de}}.
